\documentclass{beamer}
\usepackage[english]{babel}
\usetheme{CambridgeUS}

\usepackage{caption}
\captionsetup{tableposition=top,figureposition=bottom,font=small}
\usepackage{graphicx}
\usepackage{subfig}
\usepackage{grffile}
\usepackage{booktabs}
\usepackage[absolute,overlay]{textpos}
\setbeamertemplate{navigation symbols}{}
\title[]{Screen Space Ambient Occlusion \\\small A WebGL/three.js implementation}

\newsavebox{\authbox}
\sbox{\authbox}{
    \centering
    \begin{minipage}{0.45\linewidth}
        \centering\normalsize
        Ivan Prosperi
    \end{minipage}
}


\author[Ivan Prosperi]{
    \usebox{\authbox}}
\institute[]{Universit\`a degli Studi di Firenze}
\date{}
\logo{\textcolor{black}{\includegraphics[width=0.10\textwidth]{images/logo_unifi/stemma_grigio.pdf}}}


\AtBeginSection{%
    \begin{frame}
    \tableofcontents[currentsection, subsectionstyle=show/show/hide]
\end{frame}
}

\begin{document}

\begin{frame}
    \titlepage
    \centering
    \vspace*{-1.3cm}
    Computer Graphics \& 3D Project Report
\end{frame}

\begin{frame}
    \frametitle{Table of contents}
    \tableofcontents
\end{frame}

% ##################### START #####################
\section{Introduction}

\subsection{Ambient Occlusion}

\begin{frame}
\frametitle{Ambient Occlusion}
% cosa è, immagini con/senza

\end{frame}

\begin{frame}
\frametitle{Global Illumination}
% cosa è, immagini indirect light + AO vera + citazione paper AO

\end{frame}

\begin{frame}
\frametitle{Multipass Rendering}
% Dire che GI non si può ottenere e si usa multipass per approssimare
% Far vedere esempio di multipass rendering
% Scrivere tipo "used for shadow mapping, normal mapping, etc"

\end{frame}

\section{Screen Space Ambient Occlusion}

\subsection{Principles}

\begin{frame}
\frametitle{Screen Space Ambient Occlusion}
% crytek-crysis 2007

\end{frame}

% Magari mettere un altro frame qui con qualcosaltro

\subsection{Implementation Details}
% scrivere che si usa deferred rendering, scrivere tutti i gbuffer che si usano,..

% Qui metto tutte subsub con le tecniche che ho usato: tipo range check, min/max distance,.. No view space reconstruction. Quella ha subsection dedicata
\subsubsection{Range Check}
\subsubsection{Distance Constraints}

% FIXME: forse usare dei frame più che delle subsubsection per la roba sopra

\subsection{View Space Reconstruction}

\section{Tools and Software}
% threejs webgl buildpack node/npm heroku eslint

% ##################### END #####################

\end{document}